%%%%%%%%%%%%%%%%%%%%%%%%%%%%%%%%%%%%%%%%%
% Developer CV
% LaTeX Class
% Version 2.0 (12/10/23)
%
% This class originates from:
% http://www.LaTeXTemplates.com
%
% Authors:
% Omar Roldan
% Based on a template by  Jan Vorisek (jan@vorisek.me)
% Based on a template by Jan Küster (info@jankuester.com)
% Modified for LaTeX Templates by Vel (vel@LaTeXTemplates.com)
%
% License:
% The MIT License (see included LICENSE file)
%
%%%%%%%%%%%%%%%%%%%%%%%%%%%%%%%%%%%%%%%%%

%----------------------------------------------------------------------------------------
%	PACKAGES AND OTHER DOCUMENT CONFIGURATIONS
%----------------------------------------------------------------------------------------

\documentclass[9pt]{developercv} % Default font size, values from 8-12pt are recommended
\usepackage{multicol}
\usepackage[brazilian]{babel}
\usepackage[utf8]{inputenc}
\setlength{\columnsep}{0mm}
%----------------------------------------------------------------------------------------
\usepackage{lipsum}  


\begin{document}
	
	%----------------------------------------------------------------------------------------
	%	TITLE AND CONTACT INFORMATION
	%----------------------------------------------------------------------------------------
	
	\begin{minipage}[t]{0.5\textwidth} 
		\vspace{-\baselineskip} % Required for vertically aligning minipages
		
		{ \fontsize{16}{20} \textcolor{black}{\textbf{\MakeUppercase{Lucas de Souza Rodrigues}}}} % First name
		
		\vspace{6pt}
		
		{\Large Cientista de dados} % Career or current job title
	\end{minipage}
	\hfill
	\begin{minipage}[t]{0.2\textwidth} % 20% of the page width for the first row of icons
		\vspace{-\baselineskip} % Required for vertically aligning minipages
		
		% The first parameter is the FontAwesome icon name, the second is the box size and the third is the text
		%\icon{Globe}{11}{\href{http://www.google.com}{portafolio.com}}\\ 
		\icon{Phone}{11}{+55 31 99987-7405}\\
		\icon{MapMarker}{11}{Curitiba, Paraná}\\
		
	\end{minipage}
	\begin{minipage}[t]{0.27\textwidth} % 27% of the page width for the second row of icons
		\vspace{-\baselineskip} % Required for vertically aligning minipages
		
		\icon{Envelope}{11}{\href{mailto:lucasdesouzarodrigues@hotmail.com}{lucasdesouzarodrigues@hotmail.com}}\\	
		\icon{Github}{11}{\href{https://github.com/lucas-sr}{https://github.com/lucas-sr}}\\
		\icon{LinkedinSquare}{11}{\href{https://www.linkedin.com}{h/in/lucassr7/}}\\    
		
	\end{minipage}
	
	
	%----------------------------------------------------------------------------------------
	%	INTRODUCTION, SKILLS AND TECHNOLOGIES
	%----------------------------------------------------------------------------------------
	
	\begin{minipage}[t]{0.46\textwidth}
		\cvsect{Resumo}
		\vspace{-6pt}
		
		Sou uma pessoa que trabalha há mais de sete anos com projetos de inovação, ciências de dados e desenvolvimento de automações. Meu foco sempre foi bastante mão na massa e atuei em uma  ampla diversidade de projetos, indo de experimentações com processamento de áudios até gestão, implantação e evolução de uma plataforma de atendimento, passando por diferentes projetos de modelagem, previsão, classificação, chatbot e RPA. Sou brasileiro, solteiro, tenho 32 anos, nascido em Belo Horizonte residente em Curitiba. Formado em Engenharia de Controle e Automação e mestrando em Inteligência Artificial.
		
	\end{minipage}
	\hfill % Whitespace between
	\begin{minipage}[t]{0.465\textwidth}
		\cvsect{Habilidades}
		\vspace{-6pt}
		
		\begin{minipage}[t]{0.2\textwidth}
			\textbf{Programação:}
		\end{minipage}
		\hfill
		\begin{minipage}[t]{0.73\textwidth}
			Python, R, SQL, PowerBI, VBA, Git, MATLab, Azure (básico), Alteryx (básico), Datarobot, Qualtrics, LABView.  
		\end{minipage}
		\vspace{4mm}
		
		\begin{minipage}[t]{0.2\textwidth}
			\textbf{Línguas}
		\end{minipage}
		\hfill
		\begin{minipage}[t]{0.73\textwidth}
			\textbf{Português}: nativo.\\
			\textbf{Inglês}: avançado.
		\end{minipage}
		
	\end{minipage}
	%----------------------------------------------------------------------------------------
	%	EXPERIENCE
	%----------------------------------------------------------------------------------------
	\vspace{-10 pt}
	\cvsect{Experiência}
	\begin{entrylist}
		\entry
		{jan/2022 -- hoje}
		{Supervisor de Machine Learning}
		{Neodent (Grupo Straumann) - Curitiba/PR}
		{\vspace{-10pt}
			\begin{itemize}[noitemsep,topsep=0pt,parsep=0pt,partopsep=0pt, leftmargin=-1pt]
				\item Coordenador da área de Ciência de Dados e Projetos Especiais, com gestão direta de um arquiteto de dados, cientista de dados pleno e um desenvolvedor pleno.
				\item Gestão de orçamento, estruturação das atividades da área, elaboração de reportes, planejamento da estratégia.
				\item Gestão de projetos e entregas.
				\item Desenvolvimento de diferentes aplicações de dados (processamento, visualização e modelagem).
				\item Desenvolvimento de automações por RPA.
				\item Expansão de ferramentas desenvolvidas pela área para outros países, especialmente região LATAM, EUA, Canadá e China.
				\item Consultoria e apoio em projetos de dados para outros times do Brasil e internacionais.
				\item Integração de equipes multinacionais.
				\item Benchmarking de novas tecnologias e experimentação através de provas de conceito.
				\item Gestão, suporte, estruturação e evolução de um novo canal de atendimento no WhatsApp.  A plataforma lançada aumentou a satisfação dos clientes em mais de 60\%, diminuiu o tempo de atendimento em 3 horas e é responsável por mais de 300 atendimentos/dia.
				\item Reportes para C-level.
				
			\end{itemize} 
			\texttt{Python} \slashsep \texttt{SQL} \slashsep \texttt{PowerBI} \slashsep \texttt{Azure} \slashsep \texttt{VBA} \slashsep \texttt{Take Blip} \slashsep \texttt{Qualtrics}}
		\entry
		{jan/2021 -- dez/2021}
		{Especialista em Ciência de Dados}
		{Neodent (Grupo Straumann) - Curitiba/PR}
		{\vspace{-10pt}
			\begin{itemize}[noitemsep,topsep=0pt,parsep=0pt,partopsep=0pt, leftmargin=-1pt]
				\item Desenvolvimento de aplicações em Python, SQL e PowerBI  para envio de pesquisas de satisfação.
				\item Desenvolvimento de automação da coleta e cálculo de KPIs de pedidos de venda.
				\item Criação de modelos de previsão de demanda com séries temporais.
				\item Criação de modelos de classificação de duplicados nas bases de Leads e prospectos do CRM.
				\item Treinamento e aulas de Python e PowerBI e disseminação de uma cultura orientada à dados.
				\item Gestão de plataforma de atendimento de WhatsApp.
			\end{itemize} 
			\texttt{Python} \slashsep \texttt{SQL} \slashsep \texttt{PowerBI} \slashsep \texttt{Genesys Engage} \slashsep \texttt{Qualtrics}}
		\entry
		{jan/2019 -- dez/2020}
		{Consultor Sênior em Ciências de Dados}
		{EY - São Paulo/SP}
		{\vspace{-10pt}
			\begin{itemize}[noitemsep,topsep=0pt,parsep=0pt,partopsep=0pt, leftmargin=-1pt]
				\item Consultoria e execução de projetos de ciências de dados com foco em classificação, previsão e otimização. Os projetos foram executados em diferentes indústrias: petroquímica, bens de consumo e energia.
				\item Consultoria em projetos de transformação digital e assessment de maturidade digital, com foco em dados e chatbot.
				\item Elaboração de levantamento de requisitos e cálculo de esforço necessário para projetos e provas de conceito.
			\end{itemize} 
			\texttt{Python} \slashsep \texttt{Databricks} \slashsep \texttt{Azure} \slashsep \texttt{PowerBI}}
		\entry
		{jul/2017 -- \\jan/2019}
		{Analista de Tecnologia Pleno}
		{Santander - São Paulo/SP}
		{\vspace{-10pt}
			\begin{itemize}[noitemsep,topsep=0pt,parsep=0pt,partopsep=0pt, leftmargin=-1pt]
				\item Desenvolvimento de provas de conceitos e modelos de classificação de texto em R.
				\item Gestão de projeto e entrega de chatbot para atendimento interno. O chatbot trouxe retorno imediato, elevando a taxa de retenção de usuários de 50\% para cerca de 90\% com três meses de operação e feedbacks.
			\end{itemize} 
			\texttt{Python} \slashsep \texttt{R}}
		\entry
		{ago/2016 -- \\jul/2017}
		{Bolsista de Apoio Técnico II}
		{Instituto de Inovação SENAI em Engenharia de Superfícies}
		{\vspace{-10pt}
			\begin{itemize}[noitemsep,topsep=0pt,parsep=0pt,partopsep=0pt, leftmargin=-1pt]
				\item Retrofit de equipamento utilizando interface Python e arduíno.
				\item Automação das planilhas de gestão e processamento de dados de equipamentos cietíficos com VBA.
				\item Processamento e análise de arquivos de áudio em Python.
			\end{itemize} 
			\texttt{Python} \slashsep \texttt{VBA} \slashsep \texttt{Arduíno}}
	\end{entrylist}
	
	%----------------------------------------------------------------------------------------
	%	EDUCATION
	%----------------------------------------------------------------------------------------
	\vspace{-10 pt}
	\cvsect{Educação}
	\begin{entrylist}
		\entry
		{ago/2021 - fev/2024}
		{Mestrado em Engenharia Elétrica com ênfase em Intligência Artificial}
		{UFPR}
		{Tema da dissertação: “Comparativo de abordagens de aprendizado de máquina para classificação de eventos de apneia do sono”. Projeto desenvolvido em Python utilizando uma base pública de eletrocardiogramas.}
		\entry
		{mar/2010 - ago/2016}
		{Bacharelado em Engenharia de Controle e Automação}
		{UFMG}
		{Durante a graduação participei de uma iniciação científica e estágio no laboratório de Fotônica do Departamento de Física, desenvolvendo em LABView.}
		\entry
		{jan/2014 - \\dez/2014}
		{Intercâmbio}
		{Budapest University of Technology and Economics (BME)}
		{Período sanduíche como bolsista do programa Ciências Sem Fronteira em Budapeste na Hungria, cursando Engenharia Elétrica}
	\end{entrylist}
	
		%----------------------------------------------------------------------------------------
	%	Projects
	%----------------------------------------------------------------------------------------
	\cvsect{Outros}
	\begin{entrylist}
		\entry
		{2023}
		{Artigo publicado}
		{}
		{%Dummy text 
			Artigo aceito e apresentado no XVI Congresso Brasileiro de Inteligência Computacional. Título: “Análise de classificadores otimizando hiperparâmetros e reduzindo a dimensionalidade aplicados na detecção de apneia do sono”}
		\entry
		{2018}
		{Green Belt}
		{Academia Santander}
		{%Dummy text 
			Curso de Green Belt – Lean Six Sigma pela Academia Santander (curso concluído, certificação não concluída)}
		\entry
		{2014}
		{Artigo publicado}
		{}
		{%Dummy text 
			Artigo publicado no Congresso Brasileiro de Engenharia Biomédica (CBEB). Título: "Seleção de Características por Algoritmo Genético na Classificação da Cardiopatia Chagásica".}
	\end{entrylist}
	
	
	%----------------------------------------------------------------------------------------
	
\end{document}
